% АRIМА-модели. Идентификация  АRІМА-моделей. Способы определения параметров моделей. Прогнозирование в АRIМА-моделях.

\begin{definition}
    \textbf{Моделью авторегрессии и скользящего среднего $\mathrm{ARMA}\left( p, q \right)$} называется класс моделей стационарных временных рядов, задаваемый уравнением следующего вида:
    \begin{equation*}
        y_t - \phi_1 y_{t-1} - \dots - \phi_p y_{t-p} = \delta + \eps_t - \theta_1 \eps_{t-1} - \dots - \theta_q \eps_{t-q},
    \end{equation*}
    или, в записи с использованием оператора лага,
    \begin{equation*}
        \Phi\left( L \right) y_t = \delta + \Theta \left( L \right) \eps_t,
    \end{equation*}
    где $\eps_t$ -- ошибки, удовлетворяющие условиям Гаусса-Маркова.
\end{definition}

Рассмотрим нестационарный временной ряд $y_t$.
Пусть этот ряд является интегрируемым порядка $d$, то есть ряд из разностей $d$-го порядка стационарен, и ряд из разностей удовлетворяет модели $\mathrm{ARMA}\left( p, q \right)$.
Тогда исходный ряд $y_t$ называется \textbf{интегрированным процессом авторегрессии и скользящего среднего} $\mathrm{ARIMA}\left( p, d, q \right)$.
Понятно, что модель для ряда $y_t$ легко строится на основе модели для ряда $\Delta^d y_t$, поэтому задача сводится к предыдущей.

Методология Бокса-Дженкинса подбора параметров модели включает три этапа.

\paragraph{Идентификация модели.}

На первом шаге подбирается параметр $d$.
Последовательно берутся разности порядка $d$, где $d$ меняется от нуля до какой-либо границы.
В качестве параметра модели принимается наименьшее значение, при котором ряд разностей оказывается стационарным.
Стационарность ряда проверяют, применяя к ряду и графикам ACF и PACF метод пристального взгляда или тест Дики-Фуллера.
На практике обычно $d$ не превышает 2.

Для полученного стационарного ряда строятся графики выборочных ACF, PACF.
По количеству точек этих функций, статистически значимо отличных от нуля получаем несколько гипотез о параметрах $p, q$.

\paragraph{Выбор параметров модели}

Для оценки параметров модели применяется метод наименьших квадратов (в том числе нелинейный), полный или условный метод наибольшего правдоподобия.
% TODO: from the notebook @ 2020-11-23

\paragraph{Проверка адекватности модели данным}

Для проверки адекватности модели имеющимся данным используются несколько методов.
Во-первых, оценки коэффициентов должны статистически значимо отличаться от нуля.
Во-вторых, следует рассмотреть остатки регрессии $e_t = y_t - \hat{y}_t$, где $\hat{y}_t$ -- оценка соответствующего значения ряда при помощи найденных параметров регрессии.
Если модель соответствует данным, то остатки также являются белым шумом, и выборочные автокорреляции остатков распределены как $\mathcal{N}\left( 0, \frac{1}{n} \right)$, то есть близки к нулю.

Если ничто из описанного выше не помогло окончательно определиться с выбором,  % прочти наконец учебник
применяются различные эмпирические критерии, например, \textbf{информационный критерий Акаике}:
\begin{equation*}
    \mathrm{AIC} = 2 \frac{p+q}{n} + \ln \left( \frac{\sum_t e_t^2}{n} \right) \to \min
\end{equation*}

