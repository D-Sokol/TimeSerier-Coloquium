% Инвариантная мера динамической системы.

Поскольку для хаотических систем сколь угодно малые отклонения в начальных условиях приводят к экспоненциально растущей ошибке предсказания, математический аппарат, описывающий эволюцию системы набором траекторий, оказывается не слишком полезным.
Вместо этого используются понятия, взятые из теории вероятности.

Разобьем фазовое простанство $X$ на ячейки, диаметры которых не превосходят $\eps = \const$.
Для каждой ячейки вычислим долю времени, которое траектория системы проводит в этой ячейке за время $T$.
Оказывается, что предельный результат при $T \to \infty$ не зависит от начальных условий и описывается некоторой вероятностной мерой на $X$. 
Эта мера называется \textbf{инвариантной мерой}, поскольку ее вид не меняется с течением времени.

\begin{definition}[Мера]
    \textbf{Мерой} на множестве $X$ называется функция $\mu(A)$, определенная для некоторых подмножеств $A \subseteq X$ таким образом, что:
    \begin{enumerate}
    \item
        $\forall A: \mu(A) \geq 0$;
    \item
        $\forall A, B: A \cap B = \emptyset \implies \mu\left( A \cup B \right) = \mu\left( A \right) + \mu\left( B \right)$;
    \item
        $\mu\left( \emptyset \right) = 0$. \footnote{Впрочем, это равенство следует из предыдущего.}
    \end{enumerate}
    
    Множества, на которых определено значение меры, называются \textbf{измеримыми}.
    
    Если мера всего пространства $\mu\left( X \right) = 1$, то мера называется \textbf{вероятностной}.
    
    Наиболее известна мера Лебега, соответствующая длине, площади, объему или их многомерному аналогу.
\end{definition}


Рассмотрим некоторое вероятностное распределение на фазовом пространстве, заданное плотностью $p(x)$.
Найдем вид этого распредения $q(y)$ при смене координат, заданной отображением $y = f(x)$.
Отметим, что функция $q(y)$ также описывает вероятностное распределение на этом же фазовом простанстве после перемещения каждой точки $x$ в $f(x)$. Из соображений сохранения вероятности получаем:
\begin{gather}
\nonumber
    q(y) =
    \sum_{x_i: f\left( x_i \right) = y} \frac{p\left( x_i \right)}{\left| \det\left( \frac{\partial f}{\partial x} \right) \right|} =
    \int_{X} \delta\left( f(x) - y \right) p\left( x_i \right) \frac{dy}{\left| \det\left( \frac{\partial f}{\partial x} \right) \right|} =\\=
\label{6_1}
    \int_{X} \delta\left( f(x) - y \right) p\left( x_i \right) dx.
\end{gather}

Теперь рассмотрим хаотическую динамическую систему с дискретным временем, заданную уравнением $x_{n+1} = f\left( x_n \right)$.
Если в уравнение \eqref{6_1} в качестве начального распределения $p(x)$ подставить инвариантную меру, то после действия отображения вид распределения не должен измениться.
Таким образом, получаем \textbf{уравнение Перрона-Фробениуса} для дискретного времени:
\begin{equation}
\label{6_2}
    p\left( y \right) = \int \delta\left( f(x) - y \right) p\left( x \right) dx.
\end{equation}

Аналогичное уравнение для системы $\dot{x} = F(x)$ с непрерывным временем может быть получено из \eqref{6_2}.
Рассмотрим отображение $\phi^\tau(x)$, ставящее в соответствие точке $x$ точку $\hat{x}(\tau)$, где функция $\hat{x}(t)$ -- решение задачи Коши $\dot{x} = F(x)$ с начальными условиями $x(0) = x$.
Очевидно, что это преобразование также не должно менять вид плотности инвариантной меры.
Следовательно, подставляя в \eqref{6_2}, получаем:
\begin{equation*}
    p\left( y \right) = \int \delta\left( \phi^\tau(x) - y \right) p\left( x \right) dx.
\end{equation*}

Продифференцируем полученное уравнение по $\tau$:
\footnote{В общем случае $p(y)$ может зависеть от времени, в таком случае в левой части получим $\partial_\tau p(y, t+\tau)$}
\begin{gather*}
    0 =
    \int p(x) \nabla \delta\left( \phi^0\left( x \right) - y \right) \cdot d_\tau \phi^\tau(x) \big|_{\tau=0} dx.
\end{gather*}

Так как $\phi^{\tau}(x) = \hat{x}(\tau)$, то $d_\tau \phi^\tau(x) \big|_{\tau=0} = d_\tau \hat{x}(\tau) \big|_{\tau=0} = \dot{x}(0) = F(x)$, $\phi^0(x) = x$. В таком случае, по свойствам дельта-функции:
\begin{gather*}
    0 = \dots =
    \int p(x) \nabla \delta\left( x - y \right) \cdot F(x) dx =
    -\nabla\left( p(y) F(y) \right).
\end{gather*}

Таким образом, уравнение Перрона-Фробениуса свелось к хорошо изученному уравнению движения сжимаемой жидкости.

\begin{theorem}[Крылова-Боголюбова]
    Если существует хотя бы одно компактное множество $A$, инвариантное относительно $\phi^\tau(x)$, то для системы $\dot x = F(x)$ существует хотя бы одна вероятностная инвариантная мера $\mu$.
    % TODO: определение множества, инвариантного относительно отображения
\end{theorem}

\begin{theorem}[Эргодическая теорема (не мультипликативная)]
    Пусть $\mu$ -- инвариантная мера динамической системы, $g: \mathbb{R}^n \to \mathbb{R}$ -- непрерывно измеримая функция на фазовом пространстве. Тогда для почти всех $x$ по мере $\mu$ предельное среднее значение $g(x)$ равно теоретическому матожиданию:
    \begin{equation*}
        \forall t: \lim_{T \to +\infty} \frac{1}{T} \int_{t}^{t+T} g\left( x(\tau) \right) d\tau = \int g(x) \mu\left( dx \right) = \const
    \end{equation*}
    
    Однако необходимо учесть, что понятие <<почти всюду>> для меры $\mu$ может значительно отличаться от такового для меры Лебега.
    Например, для системы $\dot{x} = -x$ (<<черная дыра>>) носитель меры состоит из одной точки $x=0$; для всех остальных точек равенство может не соблюдаться.
\end{theorem}


\begin{theorem}[Теорема Пуанкаре о возвращении множеств]
    Пусть $A$ -- измеримое множество, инвариантная мера которого больше нуля.
    Тогда $\exists t > 1: \mu\left( A \cap \phi^t\left( A \right) \right) > 0$.
    Иными словами, траектории, начинающиеся в данном множестве, будут бесконечно много раз возвращаться в это же множество.
\end{theorem}

\begin{theorem}[Теорема Пуанкаре о возвращении траекторий]
    Почти все точки $x$ по мере $\mu$ устойчивы по Пуассону.
    Под устойчивостью по Пуассону понимается следующее свойство: для любой окрестности $U(x)$ $\forall T: \exists t > T: \phi^t(x) \in U$, т.е. любая траектория бесконечно много раз возвращается в окрестность своей начальной точки.
\end{theorem}

