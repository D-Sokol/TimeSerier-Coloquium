% Ляпуновский показатель. Вычисление ляпуновского показателя по временному ряду. Метод аналога. Фрейм-разложение.

Рассмотрим временной ряд (возможно, многомерный) $x_i$. Тогда для данного ряда можно построить \textbf{ряд $z$-векторов}, каждый элемент которого определяется следующим образом:
\begin{equation*}
    \vec z_i = \begin{pmatrix}
        x_i & x_{i+1} & \dots & x_{i+d-1}
    \end{pmatrix}^T,
\end{equation*}
$d = \const$ -- размерность вложения.

Построенный ряд можно использовать для оценки показателей Ляпунова. Для этого находятся пары близких $z$-векторов, после чего один из них рассматривается как возмущенная версия другого. Рассмотрим конкретные алгоритмы.

Метод аналога:
\begin{enumerate}
\item
    Для вектора <<основной траектории>> $z_i$ находим его ближайшего соседа $z_{j_1}$;
\item
    Для каждой такой пары $(i, j_1)$ траектории $z_{i+t}$ и $z_{j_1+t}$ расходятся.
    В момент $t_1$, когда норма разности между ними достигает некоторого порога $\eps_2 = \const$, вектор $z_{j_1+t_1}$ заменяется на другой вектор $z_{j_2}$, ближайший к $z_{i+t_1}$.
    Этот вектор также в какой-то момент $t_2$ заменяется на $z_{j_3}$ и так далее.
\item
    Оценка старшего показателя Ляпунова вычисляется по следующей формуле:
    \begin{equation*}
        \hat{\lambda}_1 = 
        \frac{1}{t_n - t_0} \left(
            \ln \frac{\|z_{i+t_1} - z_{j_1+t_1}\|}{\|z_{i} - z_{j_1}\|} +
            \ln \frac{\|z_{i+t_2} - z_{j_2+t_2-t_1}\|}{\|z_{i+t_1} - z_{j_2}\|} +
            \dots +
            \ln \frac{\|z_{i+t_n} - z_{j_n+t_n-t_{n-1}}\|}{\|z_{i+t_{n-1}} - z_{j_n}\|}
        \right)
    \end{equation*}
\end{enumerate}


Метод фрейм-разложения.
%TODO
