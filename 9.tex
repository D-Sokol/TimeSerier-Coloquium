% Теорема Такенса. Выбор параметров реконструкции.

\begin{theorem}[Теорема Такенса]
    Рассмотрим многообразие $M^k$ размерности $k$, являющееся подмножеством фазового пространства некоторой динамической системы $y(t) = \phi^t(y(0))$.
    Будем считать, что известный временной ряд $x_i$ образован значениями наблюдаемой функции состояния системы $x_i = h(y(i\tau)),\; y(t) \in M^k$.
    Введем еще одну функцию состояния системы $\Lambda(y): M^k \to \mathbb{R}^{m}$.
    Тогда, если $M^k$, $h$, $\phi^t$ дважды дифференцируемы, $m \geq 2k+1$ и все неподвижные точки и циклы порядка $i\tau, i < m$ обладают простыми отличными от 1 собственными значениями матрицы линеаризации, значения $h(y)$ для них различны, то \emph{в общем случае} функция $\Lambda$ задает вложение многообразия в $\mathbb{R}^{m}$.
    Это означает выполнение следующих свойств:
    \begin{enumerate}
    \item
        Функция $\Lambda$ взаимно-однозначна, обратима и обе функции $\Lambda$, $\Lambda^{-1}$ дифференцируемы;
    \item
        Для образов траекторий в $\mathbb{R}^m$ выполнены те же свойства, что для исходных траекторий, в частности, через каждую точку поверхности $S^k = \Lambda(M^k)$ проходит только одна траектория;
    \item
        На поверхности $S_k$ можно ввести динамическую систему с дискретным временем, определяющую значения $z_i = \Lambda(y(i\tau)) = \Lambda \circ \phi^{i\tau} (y(0)) = \Lambda \circ \phi^t \circ \Lambda^{-1} \left( z_0 \right) = \Psi\left( z_0 \right)$.
    \end{enumerate}
\end{theorem}

Заметим, что обе динамические системы можно рассматривать как одну динамическую систему с точностью до невырожденной замены координат.
Следовательно, все характеристики системы, не меняющиеся при таких заменах, в частности, весь спектр показателей Ляпунова, можно вычислять по экспериментальным данным.

В качестве стандартного выбора функции $\Lambda(y)$ рассматривается преобразование наблюдаемых значений в ряд $z$-векторов:
\begin{equation*}
    z_i =
    \begin{pmatrix} x_i \\ x_{i+1} \\ \dots \\ x_{i+m-1} \end{pmatrix} =
    \begin{pmatrix} h(y(i\tau)) \\ h(y((i+1)\tau)) \\ \dots \\ h(y((i+m-1)\tau)) \end{pmatrix} =
    \begin{pmatrix} h \\ h \circ \phi^\tau \\ \dots \\ h \circ \phi^{(m-1)\tau} \end{pmatrix} \left( y(i\tau) \right).
\end{equation*}

Такой выбор отображения позволяет варьировать $m$ в достаточно шировких пределах без каких-либо специальных методов, а также сводит задачу прогнозирования ряда к задаче нелинейной регрессии:
\begin{equation*}
    x_{i+1} = F\left( x_i, x_{i-1}, \dots, x_{i-m+1} \right).
\end{equation*}


Свойства динамической системы на поверхности $S^k$ зависят от свойств исходной динамической системы, от наблюдаемой функции $h$, а также от задержки $\tau$ и размерности вложения $m$.
Так как два последних параметра поддаются контролю, то естественно сформулировать задачу оптимального выбора этих параметров.

Для определения размерности вложения используется метод \textbf{ложных ближайших соседей}.
Рассмотрим пару <<близких>> векторов $z_i^{(m)}, z_j^{(m)}$ в реконструкции с размерностью вложения $m$ и соответствующие им вектора $z_i^{(m+1)}, z_j^{(m+1)}$.
Если эти вектора соответствуют близким точкам многообразия $M^k$, то при увеличении размерности на единицу они также будут близки.
В противном случае, как правило, рассматриваемая пара векторов является ложными ближайшими соседями, то есть $\| z_i^{(m)} - z_j^{(m)}\|$ мало, а $\| z_i^{(m+1)} - z_j^{(m+1)} \|$ напротив, велико.
При достижении размерности, при которой получается правильная реконструкция, количество ложных ближайших соседей резко уменьшается.

Что же касается выбора $\tau$, то считается, % ходят слухи,
что параметры $m, \tau$ являются не вполне независимыми.
Определяющим фактором, судя по всему, является величина окна реконструкции $w = (m-1)\tau$, равная временному интервалу между двумя крайними точками $z$-вектора.



