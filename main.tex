\documentclass[12pt]{article}
% Свежая версия шаблона здесь <https://www.overleaf.com/read/sqvxbnhgxxdm>


\usepackage{fontspec}
\usepackage{polyglossia}
\usepackage{geometry}
\usepackage{graphicx}
\usepackage[final]{pdfpages}
\usepackage{amsmath,amsthm,amssymb,unicode-math}
\usepackage{wrapfig,multicol,tabularx,booktabs}
\usepackage[colorlinks=true, allcolors=blue]{hyperref}
\usepackage{xcolor}
%\usepackage{slashbox} %% разделение ячеек таблиц
%\usepackage{minted} %% листинги
%\usepackage{ulem} %% зачеркивание текста


\setmainlanguage{russian}
\setotherlanguage{english}
\setkeys{russian}{babelshorthands=true}

\defaultfontfeatures{Ligatures=TeX}
\setmainfont{CMU Serif Roman}
%\setmathfont{STIX Two Math}

\newfontfamily{\cyrillicfont}{CMU Serif Roman}
\newfontfamily{\cyrillicfontrm}{CMU Serif Roman}
\newfontfamily{\cyrillicfonttt}{Courier New}
\newfontfamily{\cyrillicfontsf}{CMU Serif Roman}
\newcommand{\const}{\mathrm{const}}

%\renewcommand{\thefigure}{\thesection.\arabic{figure}}
%\renewcommand{\thetable}{\thesection.\arabic{table}}
%\numberwithin{equation}{section}
\everymath{\displaystyle}
\graphicspath{{./img/}}
\renewcommand{\qedsymbol}{$\blacksquare$}
\theoremstyle{definition}
\newtheorem{definition}{Опр.}[]
\theoremstyle{remark}
\newtheorem{statement}{Утв.}[]
\theoremstyle{plain}
\newtheorem{theorem}{Теор.}[]
\newtheorem{axiom}{Аксиома}[]
\theoremstyle{definition}
\newtheorem{example}{Пример}[]
\addto\captionsrussian{
  \renewcommand{\figurename}{Рис.}
  \renewcommand{\tablename}{Табл.}
  \renewcommand{\proofname}{$\square$}
}

\newcommand{\eps}{\varepsilon}
\renewcommand{\phi}{\varphi} % не работает
\renewcommand{\le}{\leqslant}
\renewcommand{\ge}{\geqslant}
\DeclareMathOperator{\cov}{cov}
\DeclareMathOperator{\tr}{tr}

\newcounter{QuestionID}
\newcommand{\supersection}[1]{
	\stepcounter{QuestionID}
	\section*{\arabic{QuestionID}. #1}
	\addcontentsline{toc}{section}{\arabic{QuestionID}. #1}
}


\begin{document}
    \supersection{Основные понятия теории динамических систем.}
        % Основные понятия теории динамических систем.

\begin{definition}[Динамическая система]
    \textbf{динамической системой} называется пара из фазового пространства $P$ (метрическое пространство или многообразие) и однопараметрической непрерывной или дискретной группы преобразований $P \times \mathbb{R} \to P$ или $P \times \mathbb{Z} \to P$, обозначаемой как $\phi^t(x), x \in P$.
    Для данной группы отображений должны выполняться следующие свойства:
    \begin{itemize}
    \item
        $\phi^0\left( x \right) = x$;
    \item
        $\phi^{t_1} \circ \phi^{t_2} = \phi^{t_1 + t_2}$;
    \item
        $\phi^t$ дифференцируема по времени, определена и обратима для любых корректных значений $t$.
    \end{itemize}
\end{definition}

\begin{definition}[Траектория]
    \textbf{Траекторией}, проходящей через точку $x \in P$ называется множество $\left\{ \phi^t\left( x \right) | t \in T \right\}$
\end{definition}


В случае непрерывного времени динамическую систему можно задать уравнением следующего вида:
\begin{equation*}
    \dot{x} = F(x),
    \quad
    F(x) = \left. \frac{d\phi^t(x)}{dt} \right|_{t=0}.
\end{equation*}


    \supersection{Ляпуновский показатель. Вычисление ляпуновского показателя в случае анализа систем, заданных аналитически.}
        % Ляпуновский показатель. Вычисление ляпуновского показателя в случае анализа систем, заданных аналитически.



    \supersection{Ляпуновский показатель. Вычисление ляпуновского показателя по временному ряду. Метод аналога. Фрейм-разложение.}
        % Ляпуновский показатель. Вычисление ляпуновского показателя по временному ряду. Метод аналога. Фрейм-разложение.



    \supersection{Прогнозирование на основе кластеризации. Метод Уишарта.}
        % Прогнозирование на основе кластеризации. Метод Уишарта.

Так как траектория системы прилегает к аттрактору и периодически проходит близко к самой себе в прошлом.
Разумно предположить, что если начало какого-то участка траектории близко к уже известному куску, называемому мотивом, то и продолжение этого участка совпадает с мотивом.

Как правило, в качестве мотивов выбираются центроиды кластеров в прострастве $z$-векторов, для чего необходимо уметь их кластеризовать.
Однако рассматриваемая постановка задачи омрачается тем, что неизвестно ни количество кластеров, ни даже его порядок -- десятки или миллионы, что не позволяет использовать многие из существующих алгоритмов кластеризации.

Одним из алгоритмов, которые можно использовать в данной задаче, является алгоритм Уишарта (Wishart), принимающий набор точек $V_n = \left\{ x_i \right\}_{i=1}^{n}$ и два параметра $k, h$.
Параметр $k$ представляет собой количество рассматриваемых ближайших соседей и довольно слабо влияет на конечный результат.
Параметр $h$ является пороговым значением, участвующим в определении того, считается ли некоторый набор точек <<значимым>>.
Более формально, класс $C \subseteq V_n$ называется значимым по высоте $h$, если
\begin{equation*}
    \max_{x_i, x_j \in C}
    \left| \frac{k}{n W(d_k(x_i))} - \frac{k}{n W(d_k(x_j))} \right| \geq h,
\end{equation*}
где $W(r)$ -- объем гипершара радиусом $r$, $d_k(x_i)$ -- расстояние от точки $x_i$ до ее $k$-го ближайшего соседа.
При малых значениях $h$ алгоритм возвращает множество кластеров, содержащих только одну точку, а при больших значениях $h$ практически все точки не относятся ни к одному из найденных кластеров, то есть считаются межкластерным шумом.

Результатом работы алгоритма является набор меток кластеров $w_i$, где метка $w_j = 0$ обозначает межкластерный шум. Алгоритм состоит из следующих шагов:
\begin{enumerate}
\item
    Сортируем все точки по возрастанию $d_k(x_i)$. После этой операции в начале списка находятся так называемые <<области сгущения>> -- точки, расположенные сравнительно близко к по крайней мере $k$ другим точкам;
\item
    Рассматриваем граф $G$, вершины которого будут соответствовать объектам $x_i$.
    Изначально этот граф не содержит ни одной вершины; вершины добавляются по одной в соответствии с порядком, в котором отсортированы точки.
    Если в какой-то момент времени граф содержит $n$ вершин, соответствующих $n$ первым точкам, то точка $x_{n+1}$ обрабатывается по следующим правилам:
    \begin{enumerate}
    \item
        В граф добавляется новая вершина $x_{n+1}$ (здесь и далее для простоты вершины графа обозначаются так же, как и исходные точки);
    \item
        Добавленная вершина соединяется со всеми вершинами $x_{j}, j \leq n$, для которых выполняется условие $d\left( x_{n+1}, x_j \right) \leq d_k\left( x_j \right)$ (не наоборот, так, что нельзя утверждать, что количество соседей не превосходит $k$);
    \item
        Если вершина $x_{n+1}$ оказывается изолированной, то относим эту точку к новому кластеру;
    \item
        Если вершина $x_{n+1}$ не изолирована и все ее соседи принадлежат одному и тому же кластеру, то относим эту точку к этому же кластеру;
    \item
        Если соседи $x_{n+1}$ принадлежат к различным кластерам, среди которых нет нулевого кластера (обозначающего межкластерный шум) и ровно один кластер является значимым, то присваиваем к этому значимому кластеру как саму точку $x_{n+1}$, так и всех ее соседей;
    \item
        В противном случае (то есть если соседи принадлежат к различным кластерам, причем или среди них есть нулевой, или количество значимых кластеров отлично от единицы),  % TODO: в лекции только кейс с более чем одним значимым, разве не может быть ноль?
        все кластера соседей вкупе с самой $x_{n+1}$ присоединяются к межкластерному шуму.
        % TODO: В лекциях также отмечено, что все значимые кластера считаются сфорированными и блокируются на добавление. Wakarimasen.
    \end{enumerate}
\end{enumerate}


При прогнозировании конец временного ряда сравнивается с началом каждого из обнаруженных мотивов.
Если расстояние между ними не превышает некоторого порога, то оставшаяся часть мотива дает предсказание на соответствующее число шагов вперед.
Так как количество полученных предсказаний может быть большим, то их необходимо агрегировать.
Как правило, рассматривается кластеризация множества предсказаний.
Если предсказания образуют один плотный кластер с небольшим количеством выбросов, то в качестве окончательного предсказания выбирается центроид этого кластера.
Если же обнаруживается несколько разнесенных кластеров сравнимой мощности, то точка считается непрогнозируемой.

    \supersection{Плоскость энтропия-сложность.}
        % Плоскость энтропия-сложность.

Рассмотрим временной ряд $x_t$ и соответствующий ему ряд из $z$-векторов размерности $d$:
\begin{equation*}
    \vec z_i = \begin{pmatrix}
        x_i & x_{i+1} & \dots & x_{i+d-1}
    \end{pmatrix}^T.
\end{equation*}

Каждому такому вектору сопоставляется перестановка $\sigma \in S_d$, сортирующая этот вектор, то есть такая, что
\begin{equation*}
    z_{\sigma^{-1}_0} \leq z_{\sigma^{-1}_1} \leq \dots \leq z_{\sigma^{-1}_{d-1}}.
\end{equation*}

Заметим, что в случае, когда некоторые элементы вектора одинаковы, такая перестановка не единственна, но в случае непрерывного фазового пространства этой возможностью можно пренебречь.
Также можно рассматривать не саму перестановку $\sigma$, а обратную к ней $\sigma^{-1}$. Такая замена приведет только к изменению порядка чисел, описывающих эмпирическое распределение, но не повлияет на конечный результат.
В описанном случае перестановка $\sigma$ может быть получена программно при помощи функции \texttt{argsort}, определенной во многих библиотеках.

Поскольку элементы временного ряда $x_i$ можно рассматривать как наблюдения некоторого временного процесса, $\sigma(z)$ является дискретной случайной величиной, обладающей каким-то распределением $P$.

Если рассматриваемый процесс является шумом, то $P$ -- это равномерное распределение $U$ на множестве $S_d$.
А хаотической динамической системе отвечает некое распределение $P \neq U$.
\footnote{Регулярной динамической системе также отвечает распределение, не являющееся равномерным, но к таким системам этот метод исследования обычно не применяется.}

Случайное распределение можно описать в терминах энтропии и неравновесности. Дадим соответствующие определения:
\begin{definition}[Информация по Шеннону]
    Рассмотрим произвольное распределение $P$ и событие $A$. Тогда \textbf{информацией Шеннона}, соответствующей этому событию, называется значение $I_A = f\left( \frac{1}{P(A)} \right)$, где $f$ -- некоторая функция, удовлетворяющая следующим свойствам:
    \begin{itemize}
    \item
        Функция $f$ возрастает;
    \item
        Если события $A, B$ независимы, то $I_{AB} = I_{A} + I_{B}$.
    \end{itemize}
    
    Можно показать, что этим условиям удовлетворяет только логарифмическая функция с произвольным основанием. В целях нормировки общепринято использование двоичного логарифма; единицей измерения информации в таком случае является бит. Таким образом, получаем:
    \begin{equation*}
        I_A = \log_2 \frac{1}{P(A)} = -\log_2 P(A).
    \end{equation*}
\end{definition}

\begin{definition}[Энтропия по Шеннону]
    \textbf{Энтропией по Шеннону} называется среднее количество информации, получаемое в результате одного наблюдения:
    \begin{equation*}
        H = \mean_{P} I_{\omega} = \sum_{i} I_{\omega_i} p_i = -\sum_i p_i \log p_i.
    \end{equation*}
    
    Известно, что энтропия принимает значения от $0$ до $\log N$, где $N$ -- количество элементарных событий.
    Минимальное значение принимается в случае вырожденного распределения ($p_i = \delta_{ij}$), максимальное -- в случае равномерного распределения. Также часто рассматривается \textbf{нормализованная энтропия}:
    \begin{equation*}
        \bar{H}(P) = \frac{H(P)}{H_{\text{max}}} = \frac{H(P)}{H(U)} = \frac{H(P)}{\log N} \in \left[ 0, 1 \right].
    \end{equation*}
\end{definition}

\begin{definition}[Сложность]
    Сложностью называется следующая величина:
    \begin{equation*}
        S = Q\left[ P, U \right] \cdot \bar{H}\left( P \right),
    \end{equation*}
    где неравновесность $Q$ -- некоторая метрика расстояния между распределениями, нормированная в диапазон от 0 до 1.

    Будем использовать в качестве метрики расстояния \textbf{дивергенцию Йенсена-Шеннона}:
    \begin{equation*}
        Q\left[ P_1, P_2 \right] = Q_0 \cdot \left( H\left( \frac{P_1 + P_2}{2} \right) - \frac{H(P_1) + H(P_2)}{2} \right),
    \end{equation*}
    $Q_0 = -2\left( \left( \frac{N+1}{N} \right) \ln (N+1) - 2 \ln (2N) + \ln N \right)^{-1}$ -- нормировочный коэффициент.
    % TODO: cite "Generalized statistical complexity measures Geometrical and analytical properties"
\end{definition}

Возможные положения точки $(\bar{H}, S)$ на плоскости <<энтропия-сложность>> заключены между двумя кривыми, $0 \leq H \leq 1, 0 \leq S \leq \frac{1}{2}$. При этом верхней части области соответствуют хаотические временные ряды, нижним областям -- шум.

    \supersection{Инвариантная мера динамической системы.}
        % Инвариантная мера динамической системы.

Поскольку для хаотических систем сколь угодно малые отклонения в начальных условиях приводят к экспоненциально растущей ошибке предсказания, математический аппарат, описывающий эволюцию системы набором траекторий, оказывается не слишком полезным.
Вместо этого используются понятия, взятые из теории вероятности.

Разобьем фазовое простанство $X$ на ячейки, диаметры которых не превосходят $\eps = \const$.
Для каждой ячейки вычислим долю времени, которое траектория системы проводит в этой ячейке за время $T$.
Оказывается, что предельный результат при $T \to \infty$ не зависит от начальных условий и описывается некоторой вероятностной мерой на $X$. 
Эта мера называется \textbf{инвариантной мерой}, поскольку ее вид не меняется с течением времени.

\begin{definition}[Мера]
    \textbf{Мерой} на множестве $X$ называется функция $\mu(A)$, определенная для некоторых подмножеств $A \subseteq X$ таким образом, что:
    \begin{enumerate}
    \item
        $\forall A: \mu(A) \geq 0$;
    \item
        $\forall A, B: A \cap B = \emptyset \implies \mu\left( A \cup B \right) = \mu\left( A \right) + \mu\left( B \right)$;
    \item
        $\mu\left( \emptyset \right) = 0$. \footnote{Впрочем, это равенство следует из предыдущего.}
    \end{enumerate}
    
    Множества, на которых определено значение меры, называются \textbf{измеримыми}.
    
    Если мера всего пространства $\mu\left( X \right) = 1$, то мера называется \textbf{вероятностной}.
    
    Наиболее известна мера Лебега, соответствующая длине, площади, объему или их многомерному аналогу.
\end{definition}


Рассмотрим некоторое вероятностное распределение на фазовом пространстве, заданное плотностью $p(x)$.
Найдем вид этого распредения $q(y)$ при смене координат, заданной отображением $y = f(x)$.
Отметим, что функция $q(y)$ также описывает вероятностное распределение на этом же фазовом простанстве после перемещения каждой точки $x$ в $f(x)$. Из соображений сохранения вероятности получаем:
\begin{gather}
\nonumber
    q(y) =
    \sum_{x_i: f\left( x_i \right) = y} \frac{p\left( x_i \right)}{\left| \det\left( \frac{\partial f}{\partial x} \right) \right|} =
    \int_{X} \delta\left( f(x) - y \right) p\left( x_i \right) \frac{dy}{\left| \det\left( \frac{\partial f}{\partial x} \right) \right|} =\\=
\label{6_1}
    \int_{X} \delta\left( f(x) - y \right) p\left( x_i \right) dx.
\end{gather}

Теперь рассмотрим хаотическую динамическую систему с дискретным временем, заданную уравнением $x_{n+1} = f\left( x_n \right)$.
Если в уравнение \eqref{6_1} в качестве начального распределения $p(x)$ подставить инвариантную меру, то после действия отображения вид распределения не должен измениться.
Таким образом, получаем \textbf{уравнение Перрона-Фробениуса} для дискретного времени:
\begin{equation}
\label{6_2}
    p\left( y \right) = \int \delta\left( f(x) - y \right) p\left( x \right) dx.
\end{equation}

Аналогичное уравнение для системы $\dot{x} = F(x)$ с непрерывным временем может быть получено из \eqref{6_2}.
Рассмотрим отображение $\phi^\tau(x)$, ставящее в соответствие точке $x$ точку $\hat{x}(\tau)$, где функция $\hat{x}(t)$ -- решение задачи Коши $\dot{x} = F(x)$ с начальными условиями $x(0) = x$.
Очевидно, что это преобразование также не должно менять вид плотности инвариантной меры.
Следовательно, подставляя в \eqref{6_2}, получаем:
\begin{equation*}
    p\left( y \right) = \int \delta\left( \phi^\tau(x) - y \right) p\left( x \right) dx.
\end{equation*}

Продифференцируем полученное уравнение по $\tau$:
\footnote{В общем случае $p(y)$ может зависеть от времени, в таком случае в левой части получим $\partial_\tau p(y, t+\tau)$}
\begin{gather*}
    0 =
    \int p(x) \nabla \delta\left( \phi^0\left( x \right) - y \right) \cdot d_\tau \phi^\tau(x) \big|_{\tau=0} dx.
\end{gather*}

Так как $\phi^{\tau}(x) = \hat{x}(\tau)$, то $d_\tau \phi^\tau(x) \big|_{\tau=0} = d_\tau \hat{x}(\tau) \big|_{\tau=0} = \dot{x}(0) = F(x)$, $\phi^0(x) = x$. В таком случае, по свойствам дельта-функции:
\begin{gather*}
    0 = \dots =
    \int p(x) \nabla \delta\left( x - y \right) \cdot F(x) dx =
    -\nabla\left( p(y) F(y) \right).
\end{gather*}

Таким образом, уравнение Перрона-Фробениуса свелось к хорошо изученному уравнению движения сжимаемой жидкости.

\begin{theorem}[Крылова-Боголюбова]
    Если существует хотя бы одно компактное множество $A$, инвариантное относительно $\phi^\tau(x)$, то для системы $\dot x = F(x)$ существует хотя бы одна вероятностная инвариантная мера $\mu$.
    % TODO: определение множества, инвариантного относительно отображения
\end{theorem}

\begin{theorem}[Эргодическая теорема (не мультипликативная)]
    Пусть $\mu$ -- инвариантная мера динамической системы, $g: \mathbb{R}^n \to \mathbb{R}$ -- непрерывно измеримая функция на фазовом пространстве. Тогда для почти всех $x$ по мере $\mu$ предельное среднее значение $g(x)$ равно теоретическому матожиданию:
    \begin{equation*}
        \forall t: \lim_{T \to +\infty} \frac{1}{T} \int_{t}^{t+T} g\left( x(\tau) \right) d\tau = \int g(x) \mu\left( dx \right) = \const
    \end{equation*}
    
    Однако необходимо учесть, что понятие <<почти всюду>> для меры $\mu$ может значительно отличаться от такового для меры Лебега.
    Например, для системы $\dot{x} = -x$ (<<черная дыра>>) носитель меры состоит из одной точки $x=0$; для всех остальных точек равенство может не соблюдаться.
\end{theorem}


\begin{theorem}[Теорема Пуанкаре о возвращении множеств]
    Пусть $A$ -- измеримое множество, инвариантная мера которого больше нуля.
    Тогда $\exists t > 1: \mu\left( A \cap \phi^t\left( A \right) \right) > 0$.
    Иными словами, траектории, начинающиеся в данном множестве, будут бесконечно много раз возвращаться в это же множество.
\end{theorem}

\begin{theorem}[Теорема Пуанкаре о возвращении траекторий]
    Почти все точки $x$ по мере $\mu$ устойчивы по Пуассону.
    Под устойчивостью по Пуассону понимается следующее свойство: для любой окрестности $U(x)$ $\forall T: \exists t > T: \phi^t(x) \in U$, т.е. любая траектория бесконечно много раз возвращается в окрестность своей начальной точки.
\end{theorem}


    \supersection{Мультипликативная эргодическая теорема.}
        % Мультипликативная эргодическая теорема.

Рассмотрим динамическую систему с дискретным временем $x_{n+1} = f(x_n)$, в которой каждую траекторию можно сколь угодно долго продлевать по времени в обе стороны.

\begin{theorem}[Мультипликативная эргодическая теорема, теорема Оселедца]
    Пусть для всех $k$, $x_k$ фундаментальная матрица решений линеаризованной системы $\Phi_k$, определяемая следующим образом:
    \begin{equation*}
        \Phi_{k+1} = B_k \Phi_k,\; \Phi_0 = I
    \end{equation*}
    определена и невырождена, матрица $B_k = \left.\left(\frac{\partial f_i}{\partial x_j} \right)\right|_{x_k}$ ограничена.
    
    Тогда система обладает следующими свойствами:
    \begin{enumerate}
    \item
        Почти для всех точек (по инвариантной мере $\mu$) выражение $\lim_{k \to \infty} \frac{1}{k} \ln \| B_k \|$ может принимать не более чем $n$ различных значений. Эти значения называются \textbf{показателями Ляпунова} и обозначаются $\lambda_i$.
    \item
        В каждой точке $x_k$ касательное пространство распадается в прямую сумму подпространств $R_i(x_k)$ так, что если начальный вектор возмущения $u_0 \in R_i(x_k)$, то $\lim_{k \to \pm\infty} \frac{1}{k} \ln \| u_k \| = \pm\lambda_i$.
    \item
        Подпространства, определенные в предыдущем пункте, инвариантны в том смысле, что $R_i\left( x_{k+1} \right) = B_k R_i\left( x_k \right)$.
    \end{enumerate}
\end{theorem}


    \supersection{Энтропия Колмогорова-Синая.}
        % Энтропия Колмогорова-Синая.

\begin{definition}[Энтропия Колмогорова-Синая]
    Рассмотрим динамическую систему с дискретным временем $x_{n+1} = f(x_n)$ или с дискретизованным непрерывным временем $x(t+\tau) = \phi^\tau(x(t))$.
    Разобьем фазовое пространство на непересекающиеся множества $A_i$, такие, что $\diam A_i < \eps$ по любой метрике, например, Евклидовой.
    
    Тогда введем следующую последовательность разбиений:
    \begin{equation*}
        A_{i_1 i_2 \dots i_k} = \bigcap_{j=1}^{k} f^{1-j}\left( A_{i_j} \right),
    \end{equation*}
    то есть отнесем точку $x$ к множеству $A_{i_1 i_2 \dots i_k}$, если изначально она находится в множестве $A_{i_1}$, после одного временного шага в множестве $A_{i_2}$ и так далее до $A_{i_k}$.
    Заметим, что некоторые из построенных множеств (или даже подавляющее большинство) могут быть пустыми.
    
    Для каждого разбиения глубины $k$ подсчитаем энтропию Шеннона, соответствующую инвариантной мере системы $\mu$:
    \begin{equation*}
        H\left( k \right) = - \sum_{i \in \{1,\dots,N\}^k} \mu\left( A_{*i} \right) \log \mu\left( A_{*i} \right)
    \end{equation*}
    
    Тогда \textbf{метрической энтропией} или \textbf{энтропией Колмогорова-Синая} называется предельное приращение $H(k)$ при росте $k$:
    \begin{equation*}
        K =
        \lim_{\eps \to 0} \lim_{k \to \infty} \left( H(k+1) - H(k) \right) =
        \lim_{\eps \to 0} \lim_{k \to \infty} \frac{H(k)}{k}
    \end{equation*}
\end{definition}

Для хаотических систем $K > 0$, что означает, что любое конечное количество информации о системе в начальный момент времени постепенно перестает быть хоть сколько-нибудь полезным в силу нарастания неопределенности. О таких системах говорят, что они производят информацию. Для регулярных динамических систем $K = 0$.


    \supersection{Теорема Такенса. Выбор параметров реконструкции.}
        % Теорема Такенса. Выбор параметров реконструкции.



    \supersection{Предсказательная сложность. Обучаемость и предсказуемость. Обучение в случае распределения, допускающего параметризацию.}
        % Предсказательная сложность. Обучаемость и предсказуемость. Обучение в случае распределения, допускающего параметризацию.



    \supersection{Прогнозирование в моделях регрессии. Безусловное прогнозирование. Условное прогнозирование. Прогнозирование при наличии авторегрессии ошибок.}
        % Прогнозирование в моделях регрессии. Безусловное прогнозирование. Условное прогнозирование. Прогнозирование при наличии авторегрессии ошибок.

Рассмотрим классическую регрессионную модель:
\begin{equation*}
    y = X \beta + \eps,
\end{equation*}
где $y$ -- вектор зависимых переменных, $X$ -- матрица независимых переменных, $\beta$ -- вектор параметров модели, $\eps$ -- вектор случайных ошибок, удовлетворяющих условиям Гаусса-Маркова.
Тогда под задачей прогнозирования понимается построение оценки для случайной величины $y_{n+1}$ для еще одного набора объясняющих переменных $x_{n+1}$ в предположении, что зависимая переменная удовлетворяет той же модели с теми же параметрами.

\paragraph{Безусловное прогнозирование}
В постановке безусловного прогнозирования считается, что вектор независимых переменных $x_{n+1}$ известен точно.
Тогда в качестве оценки параметров модели выберем МНК-оценки $\hat\beta, s^2$ а в качестве оценки $y_{n+1}$ следующую величину:
\begin{equation*}
    \hat{y} = x_{n+1}^T \hat\beta, \quad \hat\beta = \left( X^T X \right)^{-1} X^T y, \quad s^2 = \frac{e^T e}{n-k}.
\end{equation*}

Легко проверить, что оценка $\hat{y}$ является несмещенной и обладает наименьшей MSE среди всех линейных по $y$ несмещенных оценок:
\begin{equation*}
    \mean \left( \hat{y} - y_{n+1} \right)^2 =
    \sigma^2 \left( 1 + x_{n+1}^T \left( X^T X \right)^{-1} x_{n+1} \right).
\end{equation*}

Кроме того, если ошибки $\eps_1, \dots, \eps_{n+1}$ имеют в совокупности нормальное распределение, то нормированная ошибка
\begin{equation*}
    \frac{\hat{y} - y_{n+1}}{\sqrt{s^2 \left( 1 + x_{n+1}^T \left( X^T X \right)^{-1} x_{n+1} \right)}}
\end{equation*}
имеет распределение Стьюдента $t_{n-k}$, ($k$ -- количество независимых переменных) что позволяет построить доверительный интервал.


\paragraph{Условное прогнозирование}
В задаче условного прогнозирования считается, что вектор $x_{n+1}$ известен только с некоторой точностью, то есть фактически наблюдается вектор $z = x_{n+1} + u$, где $\mean u = 0, \var u = \sigma_u^2 I$, случайный вектор $u$ не зависит от всех ошибок $\eps_1, \dots, \eps_{n+1}$.
Тогда в качестве оценки для $y_{n+1}$ используется следующая величина:
\begin{equation*}
    \hat{y} = z^T \hat\beta.
\end{equation*}

Так как $\hat\beta$ зависит только от $\eps$, то $\hat\beta$ и $u$ независимы. Пользуясь этим фактом, легко показать, что эта оценка все еще является несмещенной:
\begin{equation*}
    \mean\left( \hat{y} - y_{n+1} \right) =
    \mean\left[\left( x_{n+1}^T + u^T \right) \hat\beta - x_{n+1}^T \beta \right] =
    x_{n+1}^T \left( \mean\hat\beta - \beta \right) + \left( \mean u \right)^T \left( \mean\hat\beta \right) =
    0,
\end{equation*}
и среднеквадратичная ошибка вычисляется как
\begin{equation*}
    \mean e^2 = \sigma^2 \left( 1 + x_{n+1}^T \left( X^T X \right)^{-1} x_{n+1} + \sigma_u^2 \tr \left( \left( X^T X \right)^{-1} \right) \right) + \sigma_u^2 \beta^T \beta.
\end{equation*}
Видно, что среднеквадратичная ошибка увеличивается и включает два новых слагаемых, пропорциональных дисперсии ошибки наблюдения $u$.
Однако аналитического выражения для построения доверительного интервала не существует.
% Такие дела, десу.


\paragraph{Авторегрессия ошибок}
Поскольку шум $\eps_t$ включает в себя все факторы, не учтенные моделью, то в реальных процессах ошибки не являются независимыми.
Поэтому рассмотрим более сложную модель, в которой ошибки образуют авторегрессионный процесс первого порядка:
\begin{equation*}
    \begin{cases}
        \eps_t = \rho \eps_{t-1} + \nu_t,\\
        y_t = x_t \cdot \beta + \eps_t,
    \end{cases}
\end{equation*}
где условиям Гаусса-Маркова удовлетворяют величины $\nu_t$, а не $\eps_t$; $|\rho| < 1$.
Более того, шум $\eps_t$ в различные моменты времени являются разными случайными величинами и могут иметь различные распределения.

Для начала предположим, что истинные параметры модели $\left( \rho, \beta \right)$ известны.
Тогда в качестве оценки $y_{n+1}$ возьмем
\begin{equation}
\label{11_1}
    \hat{y} =
    x_{n+1}^T \beta + \rho \left( y_n - x_n^T \beta \right).
\end{equation}

Легко проверить, что $e = \nu_{n+1}$, $\mean e = 0$, $\mean e^2 = \sigma_\nu^2 = \left( 1 - \rho^2 \right) \sigma_\eps^2$.
Отметим, что полученная среднеквадратичная ошибка меньше, чем в предположении некоррелированных ошибок и известного параметра $\beta$, когда $\mean e^2 = \sigma_\eps^2$.


В действительности параметры регрессии неизвестны.
Рассмотрим несколько стратегий построения оценки параметров модели:
\begin{enumerate}
\item
    Построение оценки $\left( \hat\beta, \hat\rho \right)$ по известным $x_i, y_i$, например, нелинейным методом наименьших квадратов.
\item
    Заметим, что при известном значении $\rho$ модель \eqref{11_1} можно свести к классической регрессионной модели с независимыми ошибками $\nu_t$, искусственно добавив новую независимую переменную $x_{t,k+1} = y_{t-1}$.
    Таким образом, при известном $\rho$ параметр $\beta$ может быть оценен при помощи описанных ранее методов.
    Поэтому можно перебрать значения $\rho$ в некотором промежутке, для каждого значения оценить $\beta$ и выбрать пару параметров, минимизирующих ошибку.
\item
    В дополнение к предыдущему пункту также заметим, что при известном значении $\beta$ значения $\eps_t$ также известны, поэтому параметр $\rho$ может быть оценен как параметр линейной регрессии.
    Следовательно, для оценки параметров можно применить следующий итерационный процесс:
    \begin{enumerate}
    \item
        Параметр $\rho$ оценивается любым способом, например, методом случайного угадывания;
    \item
        Считая полученную на предыдущем шаге оценку $\hat\rho$  истинным значением, вычисляем оценку параметра $\beta$;
    \item
        Считая полученную на предыдущем шаге оценку $\hat\beta$ истинным значением, вычисляем оценку параметра $\rho$ ;
    \item
        Шаги 2-3 повторяются до достижения сходимости.
    \end{enumerate}
\end{enumerate}


    \supersection{Временные ряды. Динамические модели. Единичные корни и коинтеграция. Автокорреляционная и частичная автокорреляционная функции.}
        % Временные ряды. Динамические модели. Единичные корни и коинтеграция. Автокорреляционная и частичная автокорреляционная функции.

При рассмотрении регулярных временных рядов независимые переменные ($x_t$) называются \textbf{экзогенными}, зависимые ($y_t$) -- \textbf{эндогенными}.

Стандартной моделью является модель, в которой $y_t$ линейным образом зависит от лагированных экзогенных и эндогенных переменных:
\begin{equation}
\label{12_1}
    y_t =
    \delta + \sum_{i=1}^{q} \beta_i x_{t-i} + \sum_{j=1}^{p} \alpha_j y_{y-j} + \eps_t =
    \delta + A\left( L \right) \left[ y_t \right] + B\left( L \right) \left[ x_t \right] + \eps_t,
\end{equation}
где $L$ -- оператор лага, $A, B$ -- некоторые многочлены от одной переменной, $\eps_t$ -- случайные ошибки, удовлетворяющие условиям Гаусса-Маркова (то есть они независимы, обладают нулевым матожиданием и постоянной конечной дисперсией и некоррелированы с $x_t$). В случае, когда $p > 0$, то есть присутствует зависимость от лагированных эндогенных переменных, модель \eqref{12_1} называется \textbf{динамической}.


\begin{definition}[Стационарность]
    Ряд $y_t$ называется \textbf{стационарным в узком смысле}, если для любых $m, t$ для любого набора чисел $t_1, t_2, \dots, t_m$ следующие распределения совпадают:
    \begin{equation*}
        P\left( y_{t_1}, y_{t_2}, \dots, y_{t_m} \right) =
        P\left( y_{t_1+t}, y_{t_2+t}, \dots, y_{t_m+t} \right).
    \end{equation*}
    и \textbf{стационарным в широком смысле}, если его матожидание, дисперсия и ковариационная функция не завият от времени:
    \begin{equation*}
        \mean y_{t} = \mu < \infty,\;
        \var y_t = \gamma = \const,\;
        \cov\left( y_{t_1}, y_{t_2} \right) = K\left( t_1 - t_2 \right).
    \end{equation*}
\end{definition}

Рассмотрим авторегрессионный процесс 1 порядка (AR(1)):
\begin{equation}
\label{12_2}
    y_{t} = m + \phi y_{t-1} + \eps_t
    \iff
    y_t = \left( 1 - \phi L \right)^{-1} \left( m + \eps_t \right)
\end{equation}

Если $|\phi| < 1$, то последнее равенство также можно переписать в следующем виде:
\begin{gather*}
    y_t = \dots =
    \left( 1 + \phi L + \phi^2 L^2 + \dots \right) \left( m + \eps_t \right) =\\=
    \left( 1 + \phi + \phi^2 + \dots \right) m + \eps_t + \phi \eps_{t-1} + \phi^2 \eps_{t-2} + \dots.
\end{gather*}

Можно показать, что в таком случае $\mean y_t = \frac{m}{1-\phi}, \var y_t = \frac{\sigma^2}{1-\phi^2}, \cov\left( y_t, y_{t-k} \right) = \frac{\phi^k \sigma^2}{1 - \phi^2}$, следовательно, ряд стационарен в широком смысле.

Оценку $\hat\phi$ можно получить при помощи метода наименьших квадратов:
\begin{equation*}
    \hat\phi = \frac{\sum_{t} y_{t}y_{t-1}}{\sum_s y_s^2} = \phi + \left( \sum_t \frac{y_{t-1} \eps_t}{\sum_s y_{s-1}^2} \right),
\end{equation*}
причем в рассматриваемых предположениях эта оценка является состоятельной и асимптотически нормальной: $\sqrt{n}\left( \hat\phi - \phi \right) \to \mathcal{N}\left( 0, 1-\beta^2 \right)$.

Однако если подставить в \eqref{12_2} значение $\phi = 1$, то получится процесс \textbf{случайного блуждания}, не являющийся стационарным, так как $\var y_t = \sigma^2 t$; тем более не являются стационарными процессами процессы с $|\phi| > 1$.
В общем случае аналогичная проблема возникает, если многочлен $A(L)$ из \eqref{12_1} имеет единичный корень.


\begin{definition}[интегрируемый ряд]
    Рассмотрим нестационарный ряд $x_t$.
    Назовем ряд $\Delta x_t = x_t - x_{t-1}$ рядом из разностей $x_t$ первого порядка.
    Разностью $k+1$ порядка называется ряд из разностей ряда разностей $k$-го порядка.
    Исходный ряд $x_t$ называется \textbf{интегрируемым порядка $k$, $I(k)$}, если он сам и его разности до $k-1$ порядка нестационарны, а разности $k$-го порядка стационарны.
    Стационарный ряд также считается интегрируемым рядом порядка $0$.
\end{definition}

\begin{definition}[Коинтеграция]
    Пусть два ряда $x_t, y_t$ таковы, что некоторая их линейная комбинация $y_t - \beta x_t$ стационарна.
    Тогда эти ряды называются \textbf{коинтегрированными}.
\end{definition}

В таком случае можно получить состоятельную оценку для $\beta$, применив МНК к уравнению
\begin{equation*}
    y_t = \alpha + \beta x_t + \eps_t.
\end{equation*}

Однако в отличие от обычной оценки распределение $\sqrt{n} \left( \hat\beta - \beta \right)$ не является нормальным.


\begin{definition}[Автокорреляционная функция]
    Автокорреляционной функцией (ACF) стационарного ряда $y_t$ называется коэффициент корреляции между $y_t$ и $y_{t-k}$ в зависимости от $k$:
    \begin{equation*}
        \rho_k = \frac{\cov\left( y_t, y_{t-k} \right)}{\var y_t}.
    \end{equation*}
\end{definition}

\begin{definition}[Частичная автокорреляционная функция]
    Частичной автокорреляционной функцией (PACF) стационарного ряда $y_t$ называется коэффициент корреляции между $y_t$ и $y_{t-k}$ в зависимости от $k$ при исключении влияния промежуточных значений $y_{t-1}, \dots y_{t-k+1}$. Более формально, значение $\mathrm{PACF}(k)$ вычисляется как коэффициент $\beta_k$ в МНК-решении следующего регрессионного уравнения:
    \begin{equation*}
        y_t = \beta_0 + \beta_1 y_{t-1} + \beta_2 y_{t-2} + \dots + \beta_k y_{t-k} + \eps_t
    \end{equation*}
\end{definition}






















    \supersection{Вопрос по выбору.}
        % Вопрос по выбору.

См. выше.

    \supersection{АRIМА-модели. Идентификация  АRІМА-моделей. Способы определения параметров моделей. Прогнозирование в АRIМА-моделях.}
        % АRIМА-модели. Идентификация  АRІМА-моделей. Способы определения параметров моделей. Прогнозирование в АRIМА-моделях.

\begin{definition}
    \textbf{Моделью авторегрессии и скользящего среднего $\mathrm{ARMA}\left( p, q \right)$} называется класс моделей стационарных временных рядов, задаваемый уравнением следующего вида:
    \begin{equation*}
        y_t - \phi_1 y_{t-1} - \dots - \phi_p y_{t-p} = \delta + \eps_t - \theta_1 \eps_{t-1} - \dots - \theta_q \eps_{t-q},
    \end{equation*}
    или, в записи с использованием оператора лага,
    \begin{equation*}
        \Phi\left( L \right) y_t = \delta + \Theta \left( L \right) \eps_t,
    \end{equation*}
    где $\eps_t$ -- ошибки, удовлетворяющие условиям Гаусса-Маркова.
\end{definition}

Рассмотрим нестационарный временной ряд $y_t$.
Пусть этот ряд является интегрируемым порядка $d$, то есть ряд из разностей $d$-го порядка стационарен, и ряд из разностей удовлетворяет модели $\mathrm{ARMA}\left( p, q \right)$.
Тогда исходный ряд $y_t$ называется \textbf{интегрированным процессом авторегрессии и скользящего среднего} $\mathrm{ARIMA}\left( p, d, q \right)$.
Понятно, что модель для ряда $y_t$ легко строится на основе модели для ряда $\Delta^d y_t$, поэтому задача сводится к предыдущей.

Методология Бокса-Дженкинса подбора параметров модели включает три этапа.

\paragraph{Идентификация модели.}

На первом шаге подбирается параметр $d$.
Последовательно берутся разности порядка $d$, где $d$ меняется от нуля до какой-либо границы.
В качестве параметра модели принимается наименьшее значение, при котором ряд разностей оказывается стационарным.
Стационарность ряда проверяют, применяя к ряду и графикам ACF и PACF метод пристального взгляда или тест Дики-Фуллера.
На практике обычно $d$ не превышает 2.

Для полученного стационарного ряда строятся графики выборочных ACF, PACF.
По количеству точек этих функций, статистически значимо отличных от нуля получаем несколько гипотез о параметрах $p, q$.

\paragraph{Выбор параметров модели}

Для оценки параметров модели применяется метод наименьших квадратов (в том числе нелинейный), полный или условный метод наибольшего правдоподобия.
% TODO: from the notebook @ 2020-11-23

\paragraph{Проверка адекватности модели данным}

Для проверки адекватности модели имеющимся данным используются несколько методов.
Во-первых, оценки коэффициентов должны статистически значимо отличаться от нуля.
Во-вторых, следует рассмотреть остатки регрессии $e_t = y_t - \hat{y}_t$, где $\hat{y}_t$ -- оценка соответствующего значения ряда при помощи найденных параметров регрессии.
Если модель соответствует данным, то остатки также являются белым шумом, и выборочные автокорреляции остатков распределены как $\mathcal{N}\left( 0, \frac{1}{n} \right)$, то есть близки к нулю.

Если ничто из описанного выше не помогло окончательно определиться с выбором,  % прочти наконец учебник
применяются различные эмпирические критерии, например, \textbf{информационный критерий Акаике}:
\begin{equation*}
    \mathrm{AIC} = 2 \frac{p+q}{n} + \ln \left( \frac{\sum_t e_t^2}{n} \right) \to \min
\end{equation*}


    \supersection{Временные ряды. Свойства AR(1)-процесса. Свойства автокорреляционного процесса второго порядка AR(2). Свойства процессов скользящего среднего.}
        % Временные ряды. Свойства AR(1)-процесса. Свойства автокорреляционного процесса второго порядка AR(2). Свойства процессов скользящего среднего.

Рассмотрим наиболее распространенные частные случаи моделей класса $\mathrm{ARMA}\left( p, q \right)$.

\paragraph{Авторегрессионный процесс 1 порядка $\mathrm{AR}(1) = \mathrm{ARMA}(1, 0)$}

Авторегрессионный процесс 1 порядка задается следующим уравнением:
\begin{equation*}
    y_t = \delta + \phi_1 y_{t-1} + \eps_t
\end{equation*}

Как будет показано ранее, этот процесс является стационарным при $|\phi_1| < 1$.
В таком случае временной ряд обладает следующими свойствами:
\begin{equation*}
    \mean y_t = \frac{\delta}{1 - \phi_1}, \quad
    \var y_t = \frac{\var \eps = \sigma^2}{1 - \phi_1^2}, \quad
    \rho_k = \frac{\cov\left( y_t, y_{t-k} \right)}{\var y_t} = \phi_1^k
\end{equation*}

Частичная автокорреляционная функция процесса равна 0, за исключением начальных значений $\mathrm{PACF}(0) = 1$, $\mathrm{PACF}(1) = \mathrm{ACF}(1) = \phi_1$.
Аналогичное свойство выполняется для всех моделей авторегрессионного процесса $\mathrm{PACF}_{\mathrm{AR\left( p \right)}}(k) = 0, k>p$.


\paragraph{Авторегрессионный процесс 2 порядка $\mathrm{AR}(2) = \mathrm{ARMA}(2, 0)$}

Будем считать, что $\delta = 0$, в таком случае матожидание ряда равно нулю, и модель задается следующим уравнением:
\begin{equation}
\label{15_1}
    y_t = \phi_1 y_{t-1} + \phi_2 y_{t-2} + \eps_t
\end{equation}

Вычислим значения ковариационной функции при $k > 0$:
\begin{gather}
\nonumber
    \gamma_k =
    \cov\left( y_t, y_{t-k} \right) =
    \cov\left( \phi_1 y_{t-1} + \phi_2 y_{t-2} + \eps_t, y_{t-k} \right) =
    \phi_1 \gamma_{k-1} + \phi_2 \gamma_{k-2}\\
\label{15_2}
    \rho_k = \frac{\gamma_k}{\gamma_0} = \phi_1 \rho_{k-1} + \phi_2 \rho_{k-2}
\end{gather}

Так как из свойств автокорреляционной функции $\rho_0 = 1, \rho_{-1} = \rho_{1}$, то, подставляя в уравнение \eqref{15_2} значение $k=1$, получаем:
\begin{equation*}
    \rho_1 = \frac{\phi_1}{1 - \phi_2}
    \implies
    \rho_2 = \frac{\phi_1^2}{1 - \phi_2} + \phi_2
    \implies
    \dots
\end{equation*}

Значение дисперсии найдем, умножив обе части уравнения \eqref{15_1} на $y_t$ и взяв математическое ожидание:
\begin{equation*}
    \begin{cases}
        \gamma_0 = \phi_1 \gamma_1 + \phi_2 \gamma_2 + \sigma^2,\\
        \gamma_0 \rho_1 = \gamma_1,\\
        \gamma_0 \rho_2 = \gamma_2.
    \end{cases}
\end{equation*}

Решив полученную систему уравнений, находим
\begin{equation*}
    \var y_t = \gamma_0 = \frac{\left( 1 - \phi_2 \right) \sigma^2}{\left( 1 + \phi_2 \right) \left( 1 - \phi_1 - \phi_2 \right) \left( 1 + \phi_1 - \phi_2 \right)}
\end{equation*}

Отсюда получаем, что процесс стационарен, если
\begin{equation*}
    |\phi_2| < 1, \quad
    \phi_1 + \phi_2 < 1, \quad
    \phi_2 - \phi_1 < 1.
\end{equation*}

Заметим, что эти же условия можно вывести, потребовав, чтобы корни многочлена $1 - \phi_1 L - \phi_2 L^2$ лежали вне единичного круга.

\paragraph{Процесс скользящего среднего $\mathrm{MA\left( q \right) = \mathrm{ARMA}\left( 0, q \right)}$}

Процесс скользящего среднего задается уравнением вида:
\begin{equation*}
    y_t =
    \delta + \eps_t - \theta_1 \eps_{t-1} - \dots - \theta_q \eps_{t-q} =
    \delta + \Theta(L) \eps_t.
\end{equation*}

Вычислим характеристики процесса, для удобства записи обозначив $\theta_0 = -1$:
\begin{gather*}
    \mean y_t = \delta,\\
    \var y_t = \var \delta + \var \eps_t + \theta_1^2 \eps_{t-1} + \dots + \theta_q^2 \var \eps_{t-q} = \sigma^2 \cdot \left( 1 + \sum_{i=1}^{q} \theta_i^2 \right),\\
    \gamma_k =
    \sum_{i=0}^{q} \sum_{j=0}^{q} \theta_i \theta_j \cov\left( \eps_{t-i}, \eps_{t-k-j} \right) =
    \sum_{i=0}^{q} \sum_{j=0}^{q} \theta_i \theta_j \sigma^2 \delta_{t-i, t-k-j}.
\end{gather*}

Видно, что ожидание, дисперсия и ковариации процесса не зависят от времени, то есть при любых параметрах процесс стационарен в широком смысле.
Кроме того, при $k > q$ ковариация $\gamma_k$, а значит, и $\mathrm{ACF}\left( k \right)$, равны нулю, а частичная автокорреляционная функция экспоненциально убывает.

Рассмотрим подробнее процесс $\mathrm{MA}\left( 1 \right)$:
\begin{equation*}
    y_t = \delta + \Theta\left( L \right) \eps_t,  % <-- вот он!
    \Theta\left( L \right) = 1 - \theta_1 L
\end{equation*}

Если $|\theta_1| < 1$, то оператор $\Theta\left( L \right)$ обратим, и процесс можно представить в виде авторегрессионного процесса $\mathrm{AR}\left( \infty \right)$:
\begin{align*}
    y_t &= \delta + \Theta\left( L \right) \eps_t,\\
    \Theta\left( L \right)^{-1} y_t &= \Theta\left( L \right)^{-1} \delta + \eps_t,\\
    y_t + \theta_1 y_{t-1} + \theta_1^2 y_{t-2} + \dots &= \frac{\delta}{1 - \theta_1} + \eps_t, \\
    y_t &= \frac{\delta}{1 - \theta_1} - \theta_1 y_{t-1} - \theta_1^2 y_{t-2} - \dots + \eps_t.
\end{align*}


\end{document}
