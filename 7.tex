% Мультипликативная эргодическая теорема.

Рассмотрим динамическую систему с дискретным временем $x_{n+1} = f(x_n)$, в которой каждую траекторию можно сколь угодно долго продлевать по времени в обе стороны.

\begin{theorem}[Мультипликативная эргодическая теорема, теорема Оселедца]
    Пусть для всех $k$, $x_k$ фундаментальная матрица решений линеаризованной системы $\Phi_k$, определяемая следующим образом:
    \begin{equation*}
        \Phi_{k+1} = B_k \Phi_k,\; \Phi_0 = I
    \end{equation*}
    определена и невырождена, матрица $B_k = \left.\left(\frac{\partial f_i}{\partial x_j} \right)\right|_{x_k}$ ограничена.
    
    Тогда система обладает следующими свойствами:
    \begin{enumerate}
    \item
        Почти для всех точек (по инвариантной мере $\mu$) выражение $\lim_{k \to \infty} \frac{1}{k} \ln \| B_k \|$ может принимать не более чем $n$ различных значений. Эти значения называются \textbf{показателями Ляпунова} и обозначаются $\lambda_i$.
    \item
        В каждой точке $x_k$ касательное пространство распадается в прямую сумму подпространств $R_i(x_k)$ так, что если начальный вектор возмущения $u_0 \in R_i(x_k)$, то $\lim_{k \to \pm\infty} \frac{1}{k} \ln \| u_k \| = \pm\lambda_i$.
    \item
        Подпространства, определенные в предыдущем пункте, инвариантны в том смысле, что $R_i\left( x_{k+1} \right) = B_k R_i\left( x_k \right)$.
    \end{enumerate}
\end{theorem}

