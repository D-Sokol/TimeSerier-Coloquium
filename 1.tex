% Основные понятия теории динамических систем.

\begin{definition}[Динамическая система]
    \textbf{динамической системой} называется пара из фазового пространства $P$ (метрическое пространство или многообразие) и однопараметрической непрерывной или дискретной группы преобразований $P \times \mathbb{R} \to P$ или $P \times \mathbb{Z} \to P$, обозначаемой как $\phi^t(x), x \in P$.
    Для данной группы отображений должны выполняться следующие свойства:
    \begin{itemize}
    \item
        $\phi^0\left( x \right) = x$;
    \item
        $\phi^{t_1} \circ \phi^{t_2} = \phi^{t_1 + t_2}$;
    \item
        $\phi^t$ дифференцируема по времени, определена и обратима для любых корректных значений $t$.
    \end{itemize}
\end{definition}

\begin{definition}[Траектория]
    \textbf{Траекторией}, проходящей через точку $x \in P$ называется множество $\left\{ \phi^t\left( x \right) | t \in T \right\}$
\end{definition}


В случае непрерывного времени динамическую систему можно задать уравнением следующего вида:
\begin{equation*}
    \dot{x} = F(x),
    \quad
    F(x) = \left. \frac{d\phi^t(x)}{dt} \right|_{t=0}.
\end{equation*}

